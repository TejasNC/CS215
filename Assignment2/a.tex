\documentclass{article}
\usepackage{graphicx} % Required for inserting images
\usepackage{amsmath, amssymb}

\usepackage{amsmath}
\title{CS215 Assingment2}
\author{Tejas Chaudhari}
\date{September 2024}

\begin{document}

\maketitle

\section*{Question 4}
\subsection*{Task A}

\textbf{Intuition}
\newline
We can say that the $\mathbb{E}[X]$ is the "mean" value. So, if we take we fix the values of $X$ and take very large $a$ in comparison to the $\mathbb{E}[X]$, intuitively, the probability that $X$ exceeds $a$ is very less. This is clearly seen in the given inequality.
\newline
On the other hand, if we take a to be very small if compared to the $\mathbb{E}[X]$, intuitively, the probability that $X$ exceeds $a$ should be higher. The Markov Inequality quantifies this behaviour.
\newline
\newline
\textbf{Proof}
\newline
We will prove this inequality using contradiction. Assume that the inequality does not hold. Take an  $\epsilon > 0$ such that: 
\[
\mathbb{P}(X \geq a) = \frac{\mathbb{E}[X]}{a} + \epsilon,
\]
We can split the expectation \(\mathbb{E}[X]\) in two intervals:
\[
\mathbb{E}[X] = \mathbb{E}[X \mid X < a] \cdot \mathbb{P}(X < a) + \mathbb{E}[X \mid X \geq a] \cdot \mathbb{P}(X \geq a).
\]
From the assumption, we have:
\[ \mathbb{P}(X < a) = 1 - \left( \frac{\mathbb{E}[X]}{a} + \epsilon \right).
\]
Since \(X \geq a\) in the second case, we know that:
\[
\mathbb{E}[X \mid X \geq a] \geq a.
\]
The second term expectation can be bounded as:
\[
\mathbb{E}[X \mid X \geq a] \cdot \mathbb{P}(X \geq a) \geq a \cdot \mathbb{P}(X \geq a) = a \left( \frac{\mathbb{E}[X]}{a} + \epsilon \right) = \mathbb{E}[X] + a \epsilon.
\]
For the first part,  \(\mathbb{E}[X \mid X < a] > 0\). Hence:
\[
\mathbb{E}[X \mid X < a] \cdot \mathbb{P}(X < a) > 0.
\]
Now, putting everything together, we get the total expectation:
\[
\mathbb{E}[X] > 0+ \mathbb{E}[X] + a \epsilon 
\]
\[
\mathbb{E}[X] > \mathbb{E}[X] + a \epsilon 
\]
This is a contradiction since $\epsilon > 0$.
Thus, we conclude that:
\[
\mathbb{P}(X \geq a) \leq \frac{\mathbb{E}[X]}{a}.
\]
% Proving the Chebyshev Cantelli Inequality.
\subsection*{Task B}
\
\[
\mathbb{P}((X - \mu)^2 \geq a^2) \leq \frac{\mathbb{E}[(X - \mu)^2]}{a^2}.
\]
Let us begin with transforming $\mathbb{P}[(X - \mu) \geq \tau]$ into the form on which we can apply Markov Inequality.

It is obvious that
\[
\mathbb{P}[(X - \mu) \geq \tau] \leq \mathbb{P}((X - \mu)^2 \geq \tau^2) 
\]
Take $Y=(X-\mu)$

Apply Markov Inequality to this:
\[
\mathbb{P}((X - \mu)^2 \geq \tau^2) \leq \frac{\mathbb{E}[(X - \mu)^2]}{\tau^2}
\]
\section{}
\end{document}
